\documentclass[a4paper]{report}
\title{Unity Localization Library Documentation}
\author{Piotr Paszko}
\date{2018}

\usepackage{listings}
\usepackage{xcolor}
\usepackage{indentfirst}
\usepackage{graphicx}
\usepackage{float}
\lstdefinestyle{sharpc}{language=[Sharp]C, rulecolor=\color{blue!80!black}}

\begin{document}

\maketitle

\tableofcontents

\chapter{Introduction}
\section{ULL General Description}
Unity Localization Library (ULL) is a .NET library for handling localization files. A library was created specially for Unity, with does not have any localization support.

\section{Language file}
Language file contains:
\begin{enumerate}
\item Language tag.
\item Language full name (for in-game displaying).
\item Localized text with a key.
\end{enumerate}

\subsection{Language tag}
A language tag is stored in file name:
\begin{lstlisting}[language=XML]
language_tag.lang
\end{lstlisting}

\subsection{Language full name}
A language full name is stored in first line of language file.

\subsection{Localized text with key}
Every line after language name line have to contain a localized text with a key,
created according to the pattern:

\begin{lstlisting}[language=XML]
key_name=localized_text
\end{lstlisting}

Key name and localized text can not have a "=" character inside of themselves. Every other characters are allowed.

\subsection{Language file example}
this is example of language file:
\begin{lstlisting}[language=XML]
file: en-EN.lang

English (England)
parrot=Parrot
y_parrot=Yellow Parrot
apartament=Flat
\end{lstlisting}

\section{Versioning}
ULL uses this versioning system:

major.minor.development stage.build

\begin{enumerate}
\item Major - major version number.
\item Minor - minor version number.
\item Development Stage - describes development stage of current minor version. Values meaning:
\begin{enumerate}
\item 0 - alpha,
\item 1 - beta,
\item 2 - release candidate,
\item 3 - release version.
\end{enumerate}
\item Build - build version number.
\end{enumerate}

\section{Targeted platform}
ULL is targeted for .NET 4.5.

\chapter{API Documentation}
\section{LocalizationListGenerator class}
A class used for generation a list of available language files.

\bigskip

Caution!

Class stores Exceptions handled during execution of GenerateLocalizationList() method in special queue. User can read it via NextException property.

\bigskip

\begin{tabular}{|l|l|l|}
\hline
\multicolumn{3}{|c|}{Public methods}\\
\hline
Name & Arguments & Description\\
\hline
\begin{lstlisting}[style=sharpc]
List<Localization>
GenerateLocalizationList
\end{lstlisting}&
\begin{lstlisting}[style=sharpc]
string
localizationsPath
\end{lstlisting}&
\begin{lstlisting}[style=sharpc]
Returns a list of unloaded
Localization objects.
\end{lstlisting}\\
\hline
\end{tabular}

\bigskip

\begin{tabular}{|l|l|l|}
\hline
\multicolumn{3}{|c|}{Public properties}\\
\hline
Name & Get & Set\\
\hline
\begin{lstlisting}[style=sharpc]
bool ExceptionExists
\end{lstlisting}&
\begin{lstlisting}[style=sharpc]
Returs true, if there's any Exception in
queue.
\end{lstlisting}&
-\\
\hline
\begin{lstlisting}[style=sharpc]
Exception NextException
\end{lstlisting}&
\begin{lstlisting}[style=sharpc]
Dequeues next exception and returns it.
\end{lstlisting}&
-\\
\hline
\end{tabular}

\section{Localization class}
A class that contains data from localization file.

\bigskip

\begin{tabular}{|l|l|l|}
\hline
\multicolumn{3}{|c|}{Public methods}\\
\hline
Name & Arguments & Description\\
\hline
\begin{lstlisting}[style=sharpc]
string GetLocalizationEntry
\end{lstlisting}&
\begin{lstlisting}[style=sharpc]
string id
\end{lstlisting}&
\begin{lstlisting}[style=sharpc]
Returns a localization text
identified by a key given in
argument. Returns Exception if
there's none localization with
specified key, or localization
is not loaded.
\end{lstlisting}\\
\hline
\begin{lstlisting}[style=sharpc]
void Load()
\end{lstlisting}&
-&
\begin{lstlisting}[style=sharpc]
Loads a localization from file.
Returns Exception if loading
failed.
\end{lstlisting}\\
\hline
\end{tabular}

\bigskip

\begin{tabular}{|l|l|l|}
\hline
\multicolumn{3}{|c|}{Public properties}\\
\hline
Name & Get & Set\\
\hline
\begin{lstlisting}[style=sharpc]
string Tag
\end{lstlisting}&
\begin{lstlisting}[style=sharpc]
Returns language tag.
\end{lstlisting}&
\begin{lstlisting}[style=sharpc]
private
\end{lstlisting}\\
\hline
\begin{lstlisting}[style=sharpc]
string Name
\end{lstlisting}&
\begin{lstlisting}[style=sharpc]
Returns language full name.
\end{lstlisting}&
\begin{lstlisting}[style=sharpc]
private
\end{lstlisting}\\
\hline
\begin{lstlisting}[style=sharpc]
int EntriesCount
\end{lstlisting}&
\begin{lstlisting}[style=sharpc]
Returns number of localization texts stored.
Throws Exception when localization is not
loaded.
\end{lstlisting}&
\begin{lstlisting}[style=sharpc]
private
\end{lstlisting}\\
\hline
\end{tabular}

\chapter{Using examples}
A basic example of using (without exception handling):

\begin{lstlisting}[style=sharpc]
using ULL;
using System;
using System.Collections;

//Class for managing

class LocalizationManager
{
	const string DefaultLanguageTag = "en-EN";
	
	List<Localizations> localizationsList;
	Localization selectedLocalization;
	
	public void LoadLocalizationsList()
	{
		LocalizationListGenerator localizationListGenerator =
		new LocalizationListGenerator();
		
		localizationsList = localizationListGenerator.
		GenerateLocalizationList();
			
		while(localizationsList.ExceptionExists)
		{
			Exception e = localizationsList.NextException;
			Log(e);
		}
	}
	
	public void LoadDefaultLocalization()
	{
		selectedLocalization = GetLocalization(DefaultLanguageTag);
		selectedLocalization.Load();
	}
	
	public string GetLocalizedText(string key)
	{
		return selectedLocalization.GetLocalizationEntry(key);
	}
	
	private Localization GetLocalization(string tag)
	{
		foreach(Localization l in localizationsList)
		{
			if(l.Tag == tag)
				return l;
		}
	}
}

//class using

LocalizationManager manager = new LocalizationManager();
manager.LoadLocalizationsList();
manager.LoadDefaultLocalization();

string someText = manager.GetLocalizedText("textKey");
\end{lstlisting}

\end{document}
