\documentclass[a4paper]{report}
\title{Unity Localization Library Documentation}
\author{Piotr Paszko}
\date{2018}

\usepackage{listings}
\usepackage{xcolor}
\usepackage{indentfirst}
\usepackage{graphicx}
\usepackage{float}
\lstdefinestyle{sharpc}{language=[Sharp]C, frame=lr, rulecolor=\color{blue!80!black}}

\begin{document}

\maketitle

\tableofcontents

\chapter{Introduction}
\section{ULL General Description}
Unity Localization Library (ULL) is a .NET library for handling localization files. A library was created specially for Unity, with does not have any localization support.

\section{Language file}
Language file contains:
\begin{enumerate}
\item Language tag.
\item Language full name (for in-game displaying).
\item Localized text with a key.
\end{enumerate}

\subsection{Language tag}
A language tag is stored in file name:
\begin{lstlisting}[language=XML]
language_tag.lang
\end{lstlisting}

\subsection{Language full name}
A language full name is stored in first line of language file.

\subsection{Localized text with key}
Every line after language name line have to contain a localized text with a key,
created according to the pattern:

\begin{lstlisting}[language=XML]
key_name=localized_text
\end{lstlisting}

Key name and localized text can not have a "=" character inside of themselves. Every other characters are allowed.

\subsection{Language file example}
this is example of language file:
\begin{lstlisting}[language=XML]
file: en-EN.lang

English (England)
parrot=Parrot
y_parrot=Yellow Parrot
apartament=Flat
\end{lstlisting}

\section{Versioning}
ULL uses this versioning system:

major.minor.development stage.build

\begin{enumerate}
\item Major - major version number.
\item Minor - minor version number.
\item Development Stage - describes development stage of current minor version. Values meaning:
\begin{enumerate}
\item 0 - alpha,
\item 1 - beta,
\item 2 - release candidate,
\item 3 - release version.
\end{enumerate}
\item Build - build version number.
\end{enumerate}

\section{Targeted platform}
ULL is targeted for .NET 4.5.

\chapter{API Documentation}
Need to be write after tests.

\chapter{Using examples}
Need to be write after tests.

\end{document}
